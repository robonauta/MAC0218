\documentclass[12pt]{exam}
\usepackage[brazilian]{babel}
\usepackage[utf8]{inputenc}
\usepackage{amssymb}
\usepackage{ mathrsfs }
\usepackage[fleqn]{amsmath}
\usepackage{graphicx}
\usepackage{textcomp}
\title{MAC0218 - Técnicas de Programação II \\ Proposta de Projeto}
\author{Bruno Carneiro - N.USP \\ Daniel Nunes - N.USP \\ Eduardo Rocha Laurentino - N.USP 8988212 \\ Pedro Henrique Barbosa de Almeida - N. USP 10258793}
\date{}
\begin{document}
	\clearpage
	\pagestyle{plain}
	\maketitle
	
\section{Introdução} 
	
	
	Pretendemos nos valer profundamente das metodologias ágeis de engenharia de software, cujas bases remontam ao inicio do milênio, na Terra pré-unificada, para construção de um software serviço que dialogue com as oportunidades contemporâneas de viagens cósmicas e possam servir como uma espécie de guia para todos os mochileiros espaciais humanos - de cidadãos terrestres e cientistas cósmicos a diplomatas interespaciais, sem exceção. 
	
	Trata-se de um software cujo design de interação seguirá os modelos de aplicações web da internet dos anos 2010 e, assim, será desenvolvido usando ferramentas de programação clássicas para front-end, back-end e gerenciamento de banco de dados. 

\section{Justificativa}
	
	Desde a Grande Revelação Universal de 2149, a unificação da humanidade, prevista por historiadores como Yuval Harari ainda no início do primeiro século deste milênio, durante a década de 2010, vem se consagrando como uma verdade. Não só, o advento da internet cósmico-espacial - uma das maiores conquistas tecnológicas do último século - possibilitou a descoberta de complexas sociedades inteligentes na Via Láctea, redefinindo nossas noções mais básicas de diplomacia.
	
	Nesse contexto, se por um lado os organismos replicantes foram essenciais nas primeiras etapas das viagens interplanetárias, os últimos dois anos têm consolidado a presença e, sobretudo, a relevância dos indivíduos naturalmente humanos nas viagens inter-espaciais. Contudo, o aumento quantitativo de tais viagens não foi acompanho de um qualitativo, carecendo do amparo técnico necessário para que a experiência humana fora da Terra pudesse se concretizar como efetivamente divertida, promissora e segura - independente das atribuições de cada indivíduo no Império Intergalático. 
	
    Assim, surge o projeto \textbf{"Don't Panic!"}, o primeiro grande software-serviço de operacionalidade extra-terrestre. 
	
\section{Proposta}

	A intenção aqui é que tal serviço possa ser utilizado por pessoas de diferentes funções sociais. Logo, nosso projeto prevê uma construção abrangente de serviços, de modo a contemplar diferentes demandas e cujas permissões de acesso podem ser condicionadas e definidas com especificidade para cada usuário. 
	
	Assim, o serviço \textbf{"Don't Panic!"} consistirá de três grandes seções: \\
	
    \textbf{A) EXPLORAÇÃO LIVRE DO UNIVERSO} 
    
    Serviço de mapeamento geral do universo conhecido e explorável, com informações sobre rotas de ônibus espaciais, pontos de coletas de lixo, satélites visitáveis, embaixadas e lojas Duty Free interplanetárias, e assim por diante. 
    
	Esta seção é, em geral, de livre acesso para qualquer transeunte espacial. No entanto, poderá conter sub seções específicas cujo acesso dependerá de autorização prévia, obtida ainda na Terra, a depender das motivações da viagem espacial.\\
    
    \textbf{B) DIPLOMACIA E CIÊNCIA INTERESPACIAL}
    
    Usado por diplomatas e cientistas espaciais nas suas expedições em outros planetas para retornar ao Banco de Dados da Humanidade (BDH), de modo seguro e eficiente, informações sobre suas missões e descobertas. Como o BDH é público e aberto desde a Lei Geral de Transparência Universal dos Dados de 2162, garantiremos que tal serviço ocorra respeitando tal regra de duas maneiras: 
	
	\begin{itemize}
	\item {Disponibilização do acesso aos dados brutos em formato aberto, para pesquisadores}
	\item {Portal de divulgação de informações trabalhadas em cima de tais dados, como uma espécie de newsletter, para o público em geral}
	\end{itemize}
.
	\textbf{C) SAUDADES DA MINHA TERRA}
    
    Seção individual e pessoal para os transeuntes do espaço, que servirá como uma espécie de rede social, de modo a possibilitar contato tanto entre outros transeuntes mas, principalmente, com pessoas na Terra, de familiares e amigos as entidades governamentais relevantes, garantindo estabilidade emocional e amparo social a todos os viantes do espaço. 
    
    Poderá conter espaço para troca de arquivos, envio de mensagens, agenda pessoal, formulários de requisições diversas e mais.  \\
    
	\fbox{\scalebox{1.0}{\textbf{IMPORTANTE:} detalhes das funcionalidades de cada seção serão definidas posteriormente}} 
    
\section{Metodologia}

	Seguiremos as \textbf{metodologias ágeis de desenvolvimento}, isto é, estruturaremos o projeto com foco em indivíduos e interações mais do que apenas processos e ferramentas, trabalharemos em cima deste software com documentação muito mais abrangentes e com respostas rápidas, testes contínuos e mudanças ao longo do projeto seguindo um planejamento estruturado.
	
    Em específico, usaremos as premissas \textbf{SCRUM} de metodologia ágil, com o projetos divido em ciclos (provavelmente quinzenais) - as Sprints, que representarão um conjunto de atividades a serem executadas. 
    
\section{Planejamento}
	\textbf{Até 30/03/2018:} entrega da proposta de projeto no moodle. \\    
    \textbf{Até 09/04/2018:} definição de conjunto base de funcionalidades específicas para cada seção \\
    
   	(Em construção: demais etapas serão definidos posteriormente à partir das especificações de prazo da matéria a serem atribuídas pelo professor). \\
    
\section{Disclaimer}
	
	Como pode ser visto, optou-se, neste trabalho, em realizar um projeto com contexto ficcional, para que nosso foco fosse mais voltado aos usos das ferramentas que  aprenderemos do que no valor do produto em si. 
    

	 
	
	
	
\end{document}
