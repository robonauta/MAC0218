\documentclass[12pt]{exam}
\usepackage[brazilian]{babel}
\usepackage[utf8]{inputenc}
\usepackage{amssymb}
\usepackage{ mathrsfs }
\usepackage[fleqn]{amsmath}
\usepackage{graphicx}
\usepackage{textcomp}
\title{Proposta de projeto}
\author{Pedro Henrique Barbosa de Almeida - N. USP 10258793}
\date{}
\begin{document}
	\clearpage
	\pagestyle{plain}
	\maketitle
	
	Introdução
	
	Desde a Grande Revelação Universal de 2049, a unificação da humanidade vem se consagrando como uma verdade. Nesse contexto, se por um lado os organismos replicantes foram essenciais pros primeiros passos nas viagens interplanetárias, dos últimos dois anos para cá a presença e, sobretudo, a relevância dos indivíduos humanos nas viagens cósmicas aumentaram sem que, contudo, fosse acompanhada do amparo técnico necessário para que a experiência humana fora da Terra pudesse se concretizar como efetivamente divertida, promissora e segura. E é justamente para mitigar esses problemas que surge o Don't Panic!, primeiro grande software-serviço de exploração interplanetária. 
	
	Proposta
	
	O Don't Panic! vem 
	 
	
	
	
\end{document}